%!TEX program = xelatex
\documentclass{xmu}

\begin{document}

% 电子版 / 打印版(取消注释下一行即为打印版)
\print
% 取消注释后,将在某些偶数页产生空白页,使得下一部分的内容从奇数页开始

% 取消注释后,仅使用数字作为章的编号
% \arabicchapter

% 毕业设计 / 毕业论文(取消注释下一行即为毕业设计)
% \design

% 主修 / 辅修(取消注释下一行即为辅修)
% \minor

% 标题
\title{厦门大学本科毕业论文 \LaTeX 模板}
{XMU Thesis Template by \LaTeX}

% 姓名
\author{Abel}

% 学号
\idn{XXXXXXXXXXXXXXXXXX}

% 学院
\college{XX学院}

% 专业
\subject{XXXX}

% 年级
\grade{20XX级}

% 校内指导教师
\teacher{XXX\; 教授}

% 校外指导教师(注释则不显示)
\otherteacher{XXX\; (职位)}

% 完成时间
\pubdate{二〇XX年X月XX日}

%%%%%%%% 关键词 %%%%%%%%
\keywords{XXX;XXX;XXX}
{XXXX; XXXX; XXXX}

%%%%%%%% 承诺书 %%%%%%%%

% 生成封面、承诺书
\maketitle

%%%%%%%% 致谢 %%%%%%%%

\begin{acknowledgement}
感谢互联网的开源精神,感谢每一个无私分享知识的陌生人,其中特别感谢 \href{https://github.com/chen-huaneng/xmu-template}{chen-huaneng/xmu-template} 项目提供的 \LaTeX 基础模板让我省去了很多排版中细枝末节的重复劳动。

\end{acknowledgement}

%%%%%%%% 摘要 %%%%%%%%

% 中文摘要
\begin{abstract}

\end{abstract}

% 英文摘要
\begin{enabstract}

\end{enabstract}

%%%%%%%% 目录 %%%%%%%%

% 生成中英文目录
\tableofcontents

%%%%%%%% 正文 %%%%%%%%
这里输入正文。

%%%%%%%% 参考文献 %%%%%%%%

\begin{reference}
    % 如果需要使用 bib 文件导入参考文献,则取消注释下一行
    % \bibliography{references.bib}
    % bib 文件可以通过百度学术、Google Scholar 的引用界面自动生成
    % 已自动按照 GB/T 7714-2005 设置参考文献的引用格式
\end{reference}

%%%%%%%% 附录 %%%%%%%%
\begin{appendix}
这里输入附录的内容。

\end{appendix}

\end{document}
